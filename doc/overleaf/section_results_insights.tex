\paragraph{Results and Observations:} 
%observations
The optimal value obtained by the GPR meta-model trained on 25\% of the generated data is consistent with that obtained by the GPR meta-model trained on 75\%. The optimum value $g(X)$ reported by GPR is $P_{net} \approx 5082$, which is higher than the actual optimum found in this region, which is approximately equal to $3546$.  In the case of the NN meta-model, the model with 50\% training samples shows good results where COBYLA, SLSQP and Powell converge to a point near the actual optimum. However, the Nelder-Mead optimizer did not perform well with the NN meta-model. The case study results indicate that the GPR meta-model outperforms NN for the given problem. However, it is worth noting that the computational cost of constructing the GPR meta-model is higher than NN. While GPR successfully directs the search for the optimum to the promising region, the reported optimum value differs from the actual one.

We have considered a vast design space of $150^8$ design points for the case study, where each simulation takes 130 sec. The computational cost can be huge when applying optimizers directly to the simulation model. However, deciding the number of sampling points, the meta-model, and its parameter values are critical for meta-model-based optimization. The case study demonstrates a systematic decision-making process facilitated by a visual interface. The insights in the case study can be used and applied to tackle similar problems. 

\paragraph{Insights:} The GPR model is recommended when tackling inventory optimization problems in a supply chain as it performed better than the NN meta-model for the case study. The results show that utilizing 25\% of the data samples is sufficient for the GPR meta-model for leveraging the gradient information. The number of points to sample for building the meta-model is correlated to the design space's size. And GPR meta-model works well with the small subset of design space points as training points provided that the meta-model parameters are tuned to obtain a good fit over the original data points. The meta-model parameters can be adjusted iteratively by visualizing the fit in the \textit{DataVis} tool and accepting a fit with a reasonably small MSE. Powell and SLSQP local optimizers show improved performance with GPR meta-model due to their effective utilization of the gradient information exploited by the meta-model.

%The smoothness of the approximation surface generated by GPR can be inferred from the similarity of the optimums reported by the GPR with 25\%, 50\% and 75\% training samples. It indicates that utilizing 25\% of the data samples is sufficient for the GPR meta-model for leveraging the gradient information and effectively directing the search for the optimal solution towards the promising region.
%In order to improve the accuracy of the meta-model, a larger number of points can be sampled to obtain a better fit. However, this approach would increase the time complexity of the overall optimization process. In the case study, random and regular grid sampling was utilized. However, other sampling techniques can be explored and experimented with to determine a better number of sample points. Choosing a meta-model for a given problem is a critical decision in the optimization process, and estimating the meta-model parameters is essential to obtain a good fit. 
%The optimization algorithms Powell and Nelder-Mead exhibited good performance with the GPR meta-model, while COBYLA performed well with the NN meta-model. It is necessary to conduct experiments with various optimizers when utilizing a particular meta-model, ultimately selecting the most effective one.

%This reasonably complex case study shows that multi-dimensional design space, meta-models and optimizers exploration and visualization can be done iteratively and step-by-step to find the optimal.  Validation plays a vital role in each step of this process. It is essential to have a tool that automates and assists the user with visual aids in making critical decisions. 

\paragraph{Conclusion:} We presented a detailed case study that allowed us to arrive at an approach for meta-model-based optimization for supply chains. The case study illustrated modeling of a supply chain network, estimation of the time complexity of the experiment and number of samples to build meta-models, and the choice of meta-models and optimizers to find optimum and promising regions reasonably quickly. We plan to implement this entire approach into the InventOpt tool-set. We have already built a component of the tool-set to visualize multidimensional data called \textit{DataVis}. This study serves as guidelines for design and implementation of InventOpt tool.