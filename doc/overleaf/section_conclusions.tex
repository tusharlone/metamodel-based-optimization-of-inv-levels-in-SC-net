%The conclusions for the entire paper ---
%Points:
%-- what this paper was about, what is inventopt, its purpose and current status, what did the case study show, what are the implementation plans for inventopt. It should be similar in structure to the abstract but written with a bit more concrete details and worded differently.

This paper presented the design overview and work-in-progress status of InventOpt - a Python-based open tool-set for simulating, analysing and optimizing supply chain systems. Several components of the tool-set have been implemented, and illustrated via a detailed case-study presented in this paper. The case study considers a simple inventory optimization problem which is representative of a wide class of problems relevant to supply chains and serves to justify the use of the meta-modeling based approach used by InventOpt. The detailed comparison across a set of optimization methods and model types in our case study also serves to guide us in making design decisions in the implementation of InventOpt.  

%The case study presented in the paper demonstrates the process flow of InventOpt and emphasizes the importance of design choices when using meta-model-based optimization for supply chain problems. InventOpt is a work-in-progress tool-set, that aims to aid users through simulation, visualization and optimization processes of supply chain problems. The case study serves as a foundational step in the development of this tool. In conclusion, InventOpt represents a promising solution that can provide support to supply chain practitioners and researchers.

%We presented a detailed case study that allowed us to arrive at an approach for meta-model-based optimization for supply chains. The case study illustrated modeling of a supply chain network, estimation of the time complexity of the experiment and number of samples to build meta-models, and the choice of meta-models and optimizers to find optimum and promising regions reasonably quickly. We plan to implement this entire approach into the InventOpt tool-set. We have already built a component of the tool-set to visualize multidimensional data called \textit{DataVis}. This study serves as guidelines for design and implementation of InventOpt tool.