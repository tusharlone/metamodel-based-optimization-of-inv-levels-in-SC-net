InventOpt is a Python-based, open tool-set targeted for simulation, design exploration and optimization of supply chain and inventory problems. We have currently implemented several components of the tool-set (described through a case study in Section \ref{sec:casestudy}). This section presents an overview of the planned features of the tool. InventOpt essentially consists of the following components:

%\begin{itemize}
    %\item 
    \textbf{Component Library:} An open library of parameterized component models to model the behavior of individual components in supply chains such as inventory nodes, distributors, manufacturers and demand/customer arrivals. These components can be instantiated and connected together using Python code in multiple ways to model supply chains with complex structures. They can be simulated as concurrent entities via a discrete-event simulation paradigm. The components will be implemented as Python classes with behavior routines encapsulated into a process thread implemented using Python's SimPy library to enable their concurrent simulation. The component library would allow a user to rapidly build detailed models for supply chain systems with complex interconnections. In addition, a user will be able to derive classes from the in-built component classes to describe non-standard or custom behaviors in a modular and extensible way. 
    
    %\item 
    \textbf{Routines for Accuracy-Cost Trade-off:} Once a simulation model has been created by the user, it is necessary to understand and measure the trade-off between model accuracy and computational cost. Since the model often is stochastic, it is necessary to perform multiple simulation runs to obtain an estimate of the expected value of some performance measure. The relative standard error in a performance estimate (such as the long-run average profit) reduces with the length of each simulation run and the number of independent samples. The tool will provide routines to systematically measure and display the computational cost versus accuracy trade-off curves at multiple points in the design space so that a user can predict and budget the computing time for efficient design space exploration.
    
    %\item 
    \textbf{Tool-set for Design Space Exploration}: InventOpt will provide a GUI interface that would allow a user to describe the bounds on design variables, sample the multi-dimensional design space by performing simulations at multiple points and visualize the data over a multi-dimensional space. This step is critical for gaining insight into the effect of various design variables on individual performance measures. It also assists in model validation by allowing the user to visually check for expected trends and unexpected outliers, possibly pointing to bugs in the model. Since the number of design points to evaluate can be very large, the tool-set will provide scripts that utilize modern multi-core processors for parallel, independent simulations.  The user is required to provide the objective function to the optimization tool-set as an argument. This Python function can be a wrapper function which invokes the simulation model at the design point corresponding to the objective function arguments and returns the estimated value of the performance metrics.    
     
    %\item 
    \textbf{Meta-model generation routines}: InventOpt will provide routines and a GUI interface to build and tune meta-models for optimization. The user will be able to select the meta-model type, the number of data points to build the meta-model and the split between training and test data points for measuring the extent of fit between the original data and the meta-model. The tool will also provide recommendations or default values for these choices. Once a meta-model is built, goodness of fit measures (such as sum-of-squared errors) will be reported to tune the meta model. 
    %
    Further, the tool will allow the user to tune the meta-model parameters (such as the number of hidden layers in NN or the kernel parameters used in Kriging) to generate a meta-model with a reasonably good match with the data. The GUI tool will also allow the user to visualize the measured points superimposed on the meta-model. Since the meta-model is a multidimensional surface, one possible way is to visualize it as 3D slices with two axes specified at a time and the other axis values being fixed via user input from interactive sliders. Figure \ref{fig:DataVis_slice} shows a screenshot of such a visualization interface we have implemented for this purpose. Using this interface, the user can select two dimensions at a time to visualize a 3D slice of the meta-model superimposed on the measured simulation data. For the other dimensions, the user can select a range of values or a particular value using interactive sliders, and move them to visualize how the objective surface moves with each parameter.
               
    \textbf{Guided Optimization}: Once a meta-model has been built, continuous optimizers can be applied to rapidly identify promising regions for further exploration or to directly perform an optimization over the meta-model. InventOpt will provide the user an option to select and use one of several optimization algorithms. For this, libraries such as \texttt{scipy.optimize} \cite{2020SciPy-NMeth} and parallel optimization packages such as \textit{Pymoo} \cite{pymoo} and \textit{ParMoo} \cite{chang_2022} can be utilized in the back-end.  While the case study presented in Section \ref{sec:casestudy} uses local optimizers only, the tool will provide the option of using global optimizers (such as Simulated Annealing) also. Through a simple GUI interface, InventOpt will suggest optimizers and random restart points to the user based on heuristics and the measured simulation data. It will generate convergence plots for multiple runs of the optimizers and provide various statistics for assisting in the optimization process. This process can be performed iteratively: optimizers can be used to identify good regions in the design space, which can be narrowed down for fine-grained exploration to build meta-models and optimize in these sub-regions. 
    
    To establish and validate the suitability of the meta-model based approach, we perform a detailed case study described in the next section. This case study serves as a basis for the design of InventOpt. While the optimum approach may be specific for each problem instance, we expect that 
    the problem addressed in the case study is representative of a broad class of problems in supply chain optimization and the insights gained can be translated into a general approach.
    
    
 %The case study exhaustively evaluates the model at a large number of points to have a reference solution. Then it generates solutions based on the approach described above to compare the two and gain insights into the various design choices, such as which meta-model and optimizer performs better and how many points to sample. While the optimum choice of these design parameters will vary based on the specific problem being solved, the number and type of decision variables and the variability in the model and so on, we expect that many instances of the supply chain problem will be similar to the sample case study we have taken. Thus the insights gained from this study can translate into guiding design choices for other models. We present the detailed case study in the next section. 