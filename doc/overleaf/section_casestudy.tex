
This case study considers the problem of optimizing the net average profit in a supply chain system shown in Figure \ref{fig:simple_supply_chain_net} where the decision variables for the optimization are the inventory thresholds and reorder points at various nodes.  In this case study,  we first build a modular simulation model of this system using classes from the InventOpt component library for each node and their connections, and perform basic validation. We then present a computational cost versus accuracy trade-off analysis to determine the number of simulation runs that can be reasonably performed for evaluating each point in the design space.  We then perform detailed simulation-based evaluation at points in the 8-dimensional design space along a regular grid pattern to identify the optimum points/regions with respect to a single objective function. This data serves as a reference for evaluating the meta-model based approach. We separate the measured points into a training set and a test set. Training sets containing various fractions of the total measured points are used to build two kinds of meta-models: a Neural Network based model (NN) and a Gaussian Process Regression based model (GPR). We then perform meta-model based optimization using multiple local optimization algorithms with random restarts and present our observations and insights regarding the impact of these design choices on the quality of the results and the computational effort. 
%We use open Python libraries at each step of the process. The observations from this case study serve as a basis for the design decisions of InventOpt which are summarized in the next section. 
%(namely, SimPy \cite{SimPy} for discrete-event simulation, SciPy.Optimize \cite{2020SciPy-NMeth} for optimization, and the Scikit library for building the meta-models).  

\subsection{System Model} The supply chain model is shown in Figure \ref{fig:simple_supply_chain_net} with the model parameters associated with each node shown in green color.  The arrival of customers is modeled as a Poisson process with an average arrival rate of $\lambda$.  Each arriving customer can go to one of the two retailers with probabilities $p$ and $(1-p)$ and try to purchase between 1-10 units of items of a single type. The retailers and distributors maintain an inventory of that item and follow an $<s,S>$ policy for its replenishment. In this policy, a node proceeds to replenish its inventory whenever the inventory level drops below a certain number $s$, and each refill tries to restore the inventory level to a value $S$.  The arrows in the figure indicate the direction of the flow of items and the parameters \textbf{$D$} and \textbf{$C$ }associated with each arrow indicate the \textbf{Delay} (delivery time) and \textbf{Cost} (delivery/transport cost)  respectively, of a bulk order between two nodes. Similarly, the parameter $H$ associated with each inventory node indicates the inventory \textbf{Holding cost} (per-item, per-day) at that node. For a refill order, we assume that a retailer will always prefer a distributor that results in the least delivery cost if the required number of items are available (in-stock) at both the distributors. If none of the distributors have the requested number of items in stock, the refill order is deferred to the next day. 
%
\begin{figure}[h]
  \centering
   \includegraphics[width=0.47\textwidth]{image/SupplyChain.pdf}
   %{\epsfig{file = image/SupplyChain.pdf, width = \textwidth}}
     \caption{Supply chain network example. Parameters of each node are indicated in green.}
  \label{fig:simple_supply_chain_net}
  \end{figure}
%
 We assume that each item sold to the end customer generates a profit $P$. The transport delay, cost and holding cost values are assumed to be known/fixed based on data with reasonable values assumed for this case study. The inventory threshold and reorder parameters $s,S$ at each node are assumed to be the design variables for our optimization problem, with fixed bounds.
 %and their bounds are summarized in Table \ref{tab:param_val}. 
 The goal is to find optimal values of the inventory threshold and restock levels such that the net average profit of this system is maximized.  
 %
 %\begin{table}[h]\fontsize{9pt}{10pt}\selectfont
%\begin{table}[]
%\caption{The Supply chain design parameters, their bounds and values}\label{tab:param_val}
%\centering
%
%\begin{tabular}{|c|c|c|llll|}
%\hline
%\begin{tabular}[c]{@{}c@{}}Design\\ Parameters\end{tabular} & \begin{tabular}[c]{@{}c@{}}Lower\\ Bound\end{tabular} & \begin{tabular}[c]{@{}c@{}}Upper\\ Bound\end{tabular} & \multicolumn{4}{c|}{Values}                                                          \\ \hline
%$S_{R1}$                                                    & 350                                                   & 500                                                   & \multicolumn{1}{l|}{350} & \multicolumn{1}{l|}{400} & \multicolumn{1}{l|}{450} & 500 \\
%$s_{R1}$                                                    & 100                                                   & 250                                                   & \multicolumn{1}{l|}{100} & \multicolumn{1}{l|}{150} & \multicolumn{1}{l|}{200} & 250 \\
%$S_{R2}$                                                    & 350                                                   & 500                                                   & \multicolumn{1}{l|}{350} & \multicolumn{1}{l|}{400} & \multicolumn{1}{l|}{450} & 500 \\
%$s_{R2}$                                                    & 100                                                   & 250                                                   & \multicolumn{1}{l|}{100} & \multicolumn{1}{l|}{150} & \multicolumn{1}{l|}{200} & 250 \\
%$S_{D1}$                                                    & 600                                                   & 750                                                   & \multicolumn{1}{l|}{600} & \multicolumn{1}{l|}{650} & \multicolumn{1}{l|}{700} & 750 \\
%$s_{D1}$                                                    & 350                                                   & 500                                                   & \multicolumn{1}{l|}{350} & \multicolumn{1}{l|}{400} & \multicolumn{1}{l|}{450} & 500 \\
%$S_{D2}$                                                    & 600                                                   & 750                                                   & \multicolumn{1}{l|}{600} & \multicolumn{1}{l|}{650} & \multicolumn{1}{l|}{700} & 750 \\
%$s_{D2}$                                                    & 350                                                   & 500                                                   & \multicolumn{1}{l|}{350} & \multicolumn{1}{l|}{400} & \multicolumn{1}{l|}{450} & 500 \\ \hline
%\end{tabular}


\begin{tabular}{|c|c|c|}
\hline
\begin{tabular}[c]{@{}c@{}}\textbf{Design}\\ \textbf{Parameters}\end{tabular} & \begin{tabular}[c]{@{}c@{}}\textbf{Lower}\\ \textbf{Bound}\end{tabular} & \begin{tabular}[c]{@{}c@{}}\textbf{Upper}\\ \textbf{Bound}\end{tabular} \\ \hline
$S_{R1}$                                                    & 350                                                   & 500                                                   \\ \hline
$s_{R1}$                                                    & 100                                                   & 250                                                   \\ \hline
$S_{R2}$                                                    & 350                                                   & 500                                                   \\ \hline
$s_{R2}$                                                    & 100                                                   & 250                                                   \\ \hline
$S_{D1}$                                                    & 600                                                   & 750                                                   \\ \hline
$s_{D1}$                                                    & 350                                                   & 500                                                   \\ \hline
$S_{D2}$                                                    & 600                                                   & 750                                                   \\ \hline
$s_{D2}$                                                    & 350                                                   & 500                                                   \\ \hline
\end{tabular}

%\end{table}
%
% The following segment outlines the fixed parameter values used in our study:
%\begin{itemize}
%    \item The arrival rate of the customers ($\lambda$) is fixed at 20 customers per day.
%    \item The probability of a customer buying from retailer $R_1$ is  $p = 0.5 $ (So the probability of a customer buying from $R_2$ is $(1-p)=0.5$).
%    \item The inventory holding costs for retailers are $H_{R1} = H_{R2} = 10$ units, and for distributors, are $H_{D1} = H_{D2} = 1$ units.
%    \item The delivery costs from a distributor to a retailer are listed  as $C_{11} = 5000$, $C_{12} = 6000$, $C_{21} = 7000$, and $C_{22} = 5500$ units.
%    \item The delivery delays to deliver from distributor $i$ to retailer $j$ are $D_{11} = 2$ days, $D_{12} = 3$ days, $D_{21} = 3$ days, and $D_{22}=2$ days, respectively.
%    \item Finally, the delivery costs from manufacturer to distributors are $C_{M1} = C_{M2} = 500$, and delays are $D_{M1} = 7$ days, and $D_{M2} = 8$ days, respectively. 
%    \item $P = 100$ units is the profit generated per item the retailer sells.
%\end{itemize}
%     
%
%
%The parameter values for inventory thresholds and capacities are adjustable and not predetermined. Assuming variable values for the inventory thresholds and capacities is reasonable as the retailer or distributor controls them to ensure the product's availability. \tablename~ \ref{tab:param_val} summarises the ranges of values for supply chain parameters $(S,s)$.
%The range of values these parameters can take are as follows: \\ 
%$S_{R1} \in [350,500]$, $s_{R1} \in [100,250]$, $S_{R2} \in [350,500]$, $s_{R2} \in [100,250]$, $S_{D1} \in [600,750]$, $s_{D1} \in [350,500]$, $S_{D2} \in [600,750]$, and $s_{D2} \in [350,500]$.
%
The model has been implemented by writing a basic, configurable inventory-node class and a few other classes for modeling customer arrivals and monitoring.  The concurrent behavior of these nodes is described using a \textit{Process} construct in Python's SimPy library \cite{SimPy}.  The individual nodes, such as retailers and distributors, are implemented as derived classes that are instantiated into the system model by passing their respective parameter values and interconnected to model the supply chain's structure.  These classes form a part of InventOpt's current component library. The simulation length and the randomization seed can be specified, and each simulation run generates a detailed log (for validation and insight) and an output summary consisting of performance metrics such as the average (per-day) cost of running this supply chain (which can be attributed to the holding and transport costs), the average per-day income from the sale of items and the net average profit (denoted as $P_{net}$). 

\subsection{Design Space Exploration}

%\textbf{{Computational Cost and Estimation Accuracy:}}
For a given point in the design space, say $X=(x_1,x_2,\dots,x_7)$, $f(X)$ denotes the objective function to be minimized. In our case study, $f(X)$ is the negative of the average (per-day) net profit. A single simulation run of length $L$ days provides an estimate of $f(X)$. Since the model is stochastic in nature, individual simulation runs with distinct randomization seeds will yield slightly differing outcomes. An estimate for $f(X)$ can be obtained by averaging the results across multiple (say $N$) simulation runs with distinct randomization seeds. The accuracy of a performance estimate increases with $L$ and $N$. However, the computational cost grows directly as $L\times N$. This limits the number of design points that can be evaluated, given a fixed computational budget. To assess the trade-off between the measurement accuracy and computational cost, we consider a single point in the center of the design space and plot the Relative Standard Error (RSE) in the performance estimate obtained using $N$ simulation samples, each of length $L$.  The RSE is a measure of accuracy and reduces as the square root of $N$ (${RSE}\propto {1}/{\sqrt(N)}$) since the sample average is normally distributed. Further, the RSE reduces with $L$ since the objective function is defined to be a long run average measure for a model that has a steady-state behavior. Figure \ref{fig:compute_effort} presents a plot of the RSE values measured with respect to $L$ and $N$.  

\begin{figure}[!h]
  \vspace{-0.2cm}
  \centering
   %\includegraphics[scale=0.40]{image/supply_chain_nemon.png}
   {\epsfig{file = image/rse_vs_sim_len.png, width = 0.44\textwidth}}
  %\caption{This figure illustrates the trade-off between the computational effort and the accuracy of the expected performance measure $P_{net}$.}
  \caption{Relative Std Error (RSE) in simulation-based performance estimate as a function of simulation run length $L$ and the number of samples $N$}
  \label{fig:compute_effort}  
\end{figure}

 The plot allows us to understand the accuracy versus computational cost trade-off and budget the available computational time for design spacey exploration. We select the simulation length $L$ to be 720 days and the number of simulation samples $N$ to be 60, for a reasonable accuracy (RSE = 0.34\%). This corresponds to a serial (single-threaded) computational time of \textbf{130 seconds} on a modern x86 based workstation. Thus, estimating the objective function value $f(X)$ at any given point $X$ takes approximately \textbf{130 seconds} of computational time.
 
  To gain insights on how the objective value varies with each decision parameter, we perform design space exploration by taking simulation-based performance estimates at multiple points along a regular grid in the design space. Along each axis we consider 4 equi-spaced points. This results in a total of $4^8 = 65536$ design points at which we measure $f$.  Each measurement takes 130 seconds and thus the entire exploration requires approximately \textbf{96 days} of serial computational time. We perform  simulations in parallel on a 64-core x86-based rack-server using parallel execution scripts. With a speed-up of $64$x, the design exploration could be completed in approximately two days. 
 The measured data can provide valuable insights on how each design variable affects the performance. However the data is multi-dimensional. To visualize the simulation data, we have built a GUI tool that allows the user to upload measured data as a comma-separated-values (csv) file, and observe the function values as 3D slices, by selecting two axes at a time and varying the values of other axes via interactive sliders. This allows a user to visualize trends in the objective function. Aside from incorporating this functionality into InventOpt, we have also deployed this component as a standalone, cloud-hosted free tool called \textbf{DATAvis} (\texttt{https://datavis.streamlit.app}). Figure 
 %\ref{fig:DataVis_param_select} and 
 \ref{fig:DataVis_slice} shows a screenshot of the data visualization tool. 


%\begin{figure}[h]
 %   \begin{frame}{
 %   %\vspace{-0.2cm}
 %   \centering {\epsfig{file = image/DataVis_params.png, width = 0.48\textwidth}}  
 %   }
 %   \end{frame}
 % \caption{Screenshot of \textit{DATAvis} showing selection of input and output axes from a csv data file for visualization}
 % \label{fig:DataVis_param_select}
  %\vspace{-0.1cm}
%\end{figure}

\begin{figure*}[!t]  
  \hspace*{1.5cm}
  \begin{frame}{
    %\vspace{-0.2cm}
    \centering {\epsfig{file = image/DataVis_slice.png, width = 0.85\textwidth}}
    }
  \end{frame}
  \caption{The \textit{DATAvis} tool allows a user to visualize multi-dimensional data as 3D slices by selecting two axes at a time and varying the values of other axes (at which the slice is taken) via interactive sliders. The GUI interface supports zooming/rotating the plot}
  \label{fig:DataVis_slice}
  %\vspace{-0.1cm}
\end{figure*}




\subsection{Meta-model Based Optimization} The next steps in our case study are to perform optimization and arrive at a general approach and design choices for InventOpt. These choices are with regards to the meta-model type, the number of data points needed for building a reasonably accurate meta-model and the choice of optimizer. To perform this exploration, we first find optimal points in the design space via a near-exhaustive evaluation approach. From the set of $4^8$ data points, we select good regions in the design space, and perform fine-grained sampling in its neighbourhood, comparing the objective values at each point to arrive at a point $X\rq$ which we consider as a reference solution. We use this as a means of evaluating the performance of the meta-model based approach with varying choices for the meta-model and optimizer pairs. A meta-model $g(X)$ provides an approximation to the original objective function $f(X)$, but allows significantly faster evaluation.
%
\begin{table*}[!h]\centering \fontsize{9pt}{10pt}\selectfont %\setlength{\tabcolsep}{2pt}
\caption{Results obtained using multiple optimizers over a GPR meta-model built using varying sizes of training data.}\label{tab:obtained_results}
\begin{tabular}{|lllllllllllllllll|}
\hline
\multicolumn{1}{|c|}{} & \multicolumn{1}{c|}{} & \multicolumn{1}{c|}{} & \multicolumn{1}{c|}{} & \multicolumn{8}{c|}{\textbf{optimum X found}} & \multicolumn{1}{c|}{} & \multicolumn{1}{c|}{} & \multicolumn{1}{c|}{} & \multicolumn{1}{c|}{} & \multicolumn{1}{c|}{} \\ \cline{5-12}
\multicolumn{1}{|c|}{\multirow{-2}{*}{\begin{tabular}[c]{@{}c@{}}\textbf{Meta-}\\ \textbf{model}\end{tabular}}} & \multicolumn{1}{c|}{\multirow{-2}{*}{\textbf{MSE}}} & \multicolumn{1}{c|}{\multirow{-2}{*}{\begin{tabular}[c]{@{}c@{}}\textbf{Fitting}\\ \textbf{Time}\\ \textbf{(sec)}\end{tabular}}} & \multicolumn{1}{c|}{\multirow{-2}{*}{\textbf{Optimizer}}} & \multicolumn{1}{l|}{$S_{R1}$} & \multicolumn{1}{l|}{$s_{R1}$} & \multicolumn{1}{l|}{$S_{R2}$} & \multicolumn{1}{l|}{$s_{R2}$} & \multicolumn{1}{l|}{$S_{D1}$} & \multicolumn{1}{l|}{$s_{D1}$} & \multicolumn{1}{l|}{$S_{D2}$} & \multicolumn{1}{l|}{$s_{D2}$} & \multicolumn{1}{c|}{\multirow{-2}{*}{\begin{tabular}[c]{@{}c@{}}$g(X)$ \end{tabular}}} & \multicolumn{1}{c|}{\multirow{-2}{*}{\begin{tabular}[c]{@{}c@{}}\textbf{Optimizer}\\ \textbf{Time} \\ \textbf{(sec)}\end{tabular}}} & \multicolumn{1}{c|}{\multirow{-2}{*}{\begin{tabular}[c]{@{}c@{}}\textbf{Avg num}\\ \textbf{of} $g(X)$\\ \textbf{evals}\end{tabular}}} & \multicolumn{1}{c|}{\multirow{-2}{*}{\begin{tabular}[c]{@{}c@{}}\textbf{Avg time}\\ \textbf{to compute}\\ $g(X)$\end{tabular}}} & \multicolumn{1}{c|}{\multirow{-2}{*}{$d(X,X')$}} \\ 
\multicolumn{1}{|c|}{} & \multicolumn{1}{c|}{} & \multicolumn{1}{c|}{} & \multicolumn{1}{c|}{} & \multicolumn{1}{c|}{} & \multicolumn{1}{c|}{} & \multicolumn{1}{c|}{} & \multicolumn{1}{c|}{} & \multicolumn{1}{c|}{} & \multicolumn{1}{c|}{} & \multicolumn{1}{c|}{} & \multicolumn{1}{c|}{} & \multicolumn{1}{c|}{} & \multicolumn{1}{c|}{} & \multicolumn{1}{c|}{} & \multicolumn{1}{c|}{} & \multicolumn{1}{c|}{}\\ \hline
\multicolumn{17}{|c|}{With 25\% (16,384) training data for building NN and GPR meta-models} \\ \hline
\multicolumn{1}{|l|}{NN} & \multicolumn{1}{l|}{0.0198} & \multicolumn{1}{l|}{360.21} & \multicolumn{1}{l|}{COBYLA} & \multicolumn{1}{l|}{329.8} & \multicolumn{1}{l|}{114.5} & \multicolumn{1}{l|}{352.3} & \multicolumn{1}{l|}{152.7} & \multicolumn{1}{l|}{717.7} & \multicolumn{1}{l|}{481.2} & \multicolumn{1}{l|}{717.3} & \multicolumn{1}{l|}{373.5} & \multicolumn{1}{l|}{3486.9} & \multicolumn{1}{l|}{0.87} & \multicolumn{1}{l|}{267} & \multicolumn{1}{l|}{0.0030} & 112.9 \\ \hline
\multicolumn{1}{|l|}{NN} & \multicolumn{1}{l|}{0.0198} & \multicolumn{1}{l|}{360.21} & \multicolumn{1}{l|}{Powell} & \multicolumn{1}{l|}{300.0} & \multicolumn{1}{l|}{120.9} & \multicolumn{1}{l|}{361.6} & \multicolumn{1}{l|}{180.0} & \multicolumn{1}{l|}{750.0} & \multicolumn{1}{l|}{500.0} & \multicolumn{1}{l|}{750.0} & \multicolumn{1}{l|}{369.4} & \multicolumn{1}{l|}{3713.9} & \multicolumn{1}{l|}{4.18} & \multicolumn{1}{l|}{1086} & \multicolumn{1}{l|}{0.0038} & 129.8 \\ \hline
\multicolumn{1}{|l|}{NN} & \multicolumn{1}{l|}{0.0198} & \multicolumn{1}{l|}{360.21} & \multicolumn{1}{l|}{SLSQP} & \multicolumn{1}{l|}{317.2} & \multicolumn{1}{l|}{134.3} & \multicolumn{1}{l|}{419.7} & \multicolumn{1}{l|}{239.2} & \multicolumn{1}{l|}{750.0} & \multicolumn{1}{l|}{500.0} & \multicolumn{1}{l|}{750.0} & \multicolumn{1}{l|}{350.0} & \multicolumn{1}{l|}{3796.4} & \multicolumn{1}{l|}{2.51} & \multicolumn{1}{l|}{697} & \multicolumn{1}{l|}{0.0036} & 196.3 \\ \hline
\multicolumn{1}{|l|}{NN} & \multicolumn{1}{l|}{0.0198} & \multicolumn{1}{l|}{360.21} & \multicolumn{1}{l|}{Nelder-Mead} & \multicolumn{1}{l|}{305.9} & \multicolumn{1}{l|}{125.7} & \multicolumn{1}{l|}{508.6} & \multicolumn{1}{l|}{\cellcolor[HTML]{F8CBAD}{\color[HTML]{FF0000} 270.4}} & \multicolumn{1}{l|}{\cellcolor[HTML]{F8CBAD}{\color[HTML]{FF0000} 477.0}} & \multicolumn{1}{l|}{499.7} & \multicolumn{1}{l|}{\cellcolor[HTML]{F8CBAD}{\color[HTML]{FF0000} 1269.1}} & \multicolumn{1}{l|}{493.1} & \multicolumn{1}{l|}{3997.2} & \multicolumn{1}{l|}{5.53} & \multicolumn{1}{l|}{1578} & \multicolumn{1}{l|}{0.0035} & - \\ \hline
\multicolumn{1}{|l|}{GPR} & \multicolumn{1}{l|}{0.0008} & \multicolumn{1}{l|}{1081.76} & \multicolumn{1}{l|}{COBYLA} & \multicolumn{1}{l|}{317.5} & \multicolumn{1}{l|}{125.2} & \multicolumn{1}{l|}{319.3} & \multicolumn{1}{l|}{123.5} & \multicolumn{1}{l|}{621.7} & \multicolumn{1}{l|}{372.3} & \multicolumn{1}{l|}{732.1} & \multicolumn{1}{l|}{476.6} & \multicolumn{1}{l|}{4203.5} & \multicolumn{1}{l|}{8.49} & \multicolumn{1}{l|}{355} & \multicolumn{1}{l|}{0.0239} & 156.5 \\ \hline
\multicolumn{1}{|l|}{GPR} & \multicolumn{1}{l|}{0.0008} & \multicolumn{1}{l|}{1051.24} & \multicolumn{1}{l|}{Powell} & \multicolumn{1}{l|}{320.9} & \multicolumn{1}{l|}{124.3} & \multicolumn{1}{l|}{320.8} & \multicolumn{1}{l|}{126.3} & \multicolumn{1}{l|}{730.4} & \multicolumn{1}{l|}{478.2} & \multicolumn{1}{l|}{731.0} & \multicolumn{1}{l|}{477.9} & \multicolumn{1}{l|}{4327.5} & \multicolumn{1}{l|}{10.29} & \multicolumn{1}{l|}{346} & \multicolumn{1}{l|}{0.0297} & 10.0 \\ \hline
\multicolumn{1}{|l|}{GPR} & \multicolumn{1}{l|}{0.0008} & \multicolumn{1}{l|}{858.31} & \multicolumn{1}{l|}{SLSQP} & \multicolumn{1}{l|}{317.5} & \multicolumn{1}{l|}{125.2} & \multicolumn{1}{l|}{319.3} & \multicolumn{1}{l|}{123.5} & \multicolumn{1}{l|}{621.7} & \multicolumn{1}{l|}{372.3} & \multicolumn{1}{l|}{732.1} & \multicolumn{1}{l|}{476.6} & \multicolumn{1}{l|}{4203.5} & \multicolumn{1}{l|}{5.52} & \multicolumn{1}{l|}{172} & \multicolumn{1}{l|}{0.0321} & 156.5 \\ \hline
\multicolumn{1}{|l|}{GPR} & \multicolumn{1}{l|}{0.0008} & \multicolumn{1}{l|}{1070.81} & \multicolumn{1}{l|}{Nelder-Mead} & \multicolumn{1}{l|}{318.7} & \multicolumn{1}{l|}{129.3} & \multicolumn{1}{l|}{319.0} & \multicolumn{1}{l|}{125.5} & \multicolumn{1}{l|}{732.4} & \multicolumn{1}{l|}{374.4} & \multicolumn{1}{l|}{731.9} & \multicolumn{1}{l|}{478.7} & \multicolumn{1}{l|}{4296.1} & \multicolumn{1}{l|}{25.31} & \multicolumn{1}{l|}{861} & \multicolumn{1}{l|}{0.0294} & 105.9 \\ \hline
\multicolumn{17}{|c|}{With 50\% (32,768) training data for building NN and GPR meta-models} \\ \hline
\multicolumn{1}{|l|}{NN} & \multicolumn{1}{l|}{0.0351} & \multicolumn{1}{l|}{1135.51} & \multicolumn{1}{l|}{COBYLA} & \multicolumn{1}{l|}{366.5} & \multicolumn{1}{l|}{113.8} & \multicolumn{1}{l|}{436.9} & \multicolumn{1}{l|}{232.1} & \multicolumn{1}{l|}{706.5} & \multicolumn{1}{l|}{409.6} & \multicolumn{1}{l|}{694.0} & \multicolumn{1}{l|}{459.6} & \multicolumn{1}{l|}{3281.8} & \multicolumn{1}{l|}{1.245} & \multicolumn{1}{l|}{420} & \multicolumn{1}{l|}{0.0030} & 184.6 \\ \hline
\multicolumn{1}{|l|}{NN} & \multicolumn{1}{l|}{0.0351} & \multicolumn{1}{l|}{1135.51} & \multicolumn{1}{l|}{Powell} & \multicolumn{1}{l|}{301.8} & \multicolumn{1}{l|}{117.3} & \multicolumn{1}{l|}{300.0} & \multicolumn{1}{l|}{120.3} & \multicolumn{1}{l|}{750.0} & \multicolumn{1}{l|}{500.0} & \multicolumn{1}{l|}{750.0} & \multicolumn{1}{l|}{500.0} & \multicolumn{1}{l|}{3618.6} & \multicolumn{1}{l|}{4.108} & \multicolumn{1}{l|}{1000} & \multicolumn{1}{l|}{0.0041} & 51.1 \\ \hline
\multicolumn{1}{|l|}{NN} & \multicolumn{1}{l|}{0.0351} & \multicolumn{1}{l|}{1135.51} & \multicolumn{1}{l|}{SLSQP} & \multicolumn{1}{l|}{300.0} & \multicolumn{1}{l|}{117.3} & \multicolumn{1}{l|}{300.0} & \multicolumn{1}{l|}{119.8} & \multicolumn{1}{l|}{750.0} & \multicolumn{1}{l|}{500.0} & \multicolumn{1}{l|}{750.0} & \multicolumn{1}{l|}{500.0} & \multicolumn{1}{l|}{3620.2} & \multicolumn{1}{l|}{2.341} & \multicolumn{1}{l|}{617} & \multicolumn{1}{l|}{0.0038} & 51.8 \\ \hline
\multicolumn{1}{|l|}{NN} & \multicolumn{1}{l|}{0.0351} & \multicolumn{1}{l|}{1135.51} & \multicolumn{1}{l|}{Nelder-Mead} & \multicolumn{8}{l|}{optimum X found out of the bounds} & \multicolumn{1}{l|}{-} & \multicolumn{1}{l|}{5.479} & \multicolumn{1}{l|}{1600} & \multicolumn{1}{l|}{0.0034} & - \\ \hline
\multicolumn{1}{|l|}{GPR} & \multicolumn{1}{l|}{0.0015} & \multicolumn{1}{l|}{5679.88} & \multicolumn{1}{l|}{COBYLA} & \multicolumn{1}{l|}{320.1} & \multicolumn{1}{l|}{127.5} & \multicolumn{1}{l|}{319.5} & \multicolumn{1}{l|}{124.0} & \multicolumn{1}{l|}{730.3} & \multicolumn{1}{l|}{371.2} & \multicolumn{1}{l|}{732.0} & \multicolumn{1}{l|}{477.6} & \multicolumn{1}{l|}{4736.3} & \multicolumn{1}{l|}{23.09} & \multicolumn{1}{l|}{318} & \multicolumn{1}{l|}{0.0726} & 109.0 \\ \hline
\multicolumn{1}{|l|}{GPR} & \multicolumn{1}{l|}{0.0015} & \multicolumn{1}{l|}{6044.31} & \multicolumn{1}{l|}{Powell} & \multicolumn{1}{l|}{321.3} & \multicolumn{1}{l|}{124.8} & \multicolumn{1}{l|}{319.5} & \multicolumn{1}{l|}{124.2} & \multicolumn{1}{l|}{730.9} & \multicolumn{1}{l|}{479.7} & \multicolumn{1}{l|}{731.2} & \multicolumn{1}{l|}{478.0} & \multicolumn{1}{l|}{4780.6} & \multicolumn{1}{l|}{22.09} & \multicolumn{1}{l|}{348} & \multicolumn{1}{l|}{0.0635} & 10.0 \\ \hline
\multicolumn{1}{|l|}{GPR} & \multicolumn{1}{l|}{0.0015} & \multicolumn{1}{l|}{6147.85} & \multicolumn{1}{l|}{SLSQP} & \multicolumn{1}{l|}{320.1} & \multicolumn{1}{l|}{127.5} & \multicolumn{1}{l|}{319.5} & \multicolumn{1}{l|}{124.0} & \multicolumn{1}{l|}{730.3} & \multicolumn{1}{l|}{371.2} & \multicolumn{1}{l|}{732.0} & \multicolumn{1}{l|}{477.6} & \multicolumn{1}{l|}{4736.3} & \multicolumn{1}{l|}{8.74} & \multicolumn{1}{l|}{149} & \multicolumn{1}{l|}{0.0587} & 109.0 \\ \hline
\multicolumn{1}{|l|}{GPR} & \multicolumn{1}{l|}{0.0015} & \multicolumn{1}{l|}{5774.36} & \multicolumn{1}{l|}{Nelder-Mead} & \multicolumn{1}{l|}{320.1} & \multicolumn{1}{l|}{127.5} & \multicolumn{1}{l|}{319.5} & \multicolumn{1}{l|}{124.0} & \multicolumn{1}{l|}{730.3} & \multicolumn{1}{l|}{371.2} & \multicolumn{1}{l|}{732.0} & \multicolumn{1}{l|}{477.6} & \multicolumn{1}{l|}{4736.3} & \multicolumn{1}{l|}{86.65} & \multicolumn{1}{l|}{878} & \multicolumn{1}{l|}{0.0987} & 109.0 \\ \hline
\multicolumn{17}{|c|}{With 75\% (49,152) training data for building NN and GPR meta-models} \\ \hline
\multicolumn{1}{|l|}{NN} & \multicolumn{1}{l|}{0.0469} & \multicolumn{1}{l|}{1524.81} & \multicolumn{1}{l|}{COBYLA} & \multicolumn{1}{l|}{402.5} & \multicolumn{1}{l|}{183.8} & \multicolumn{1}{l|}{332.4} & \multicolumn{1}{l|}{165.0} & \multicolumn{1}{l|}{732.2} & \multicolumn{1}{l|}{487.0} & \multicolumn{1}{l|}{737.4} & \multicolumn{1}{l|}{420.7} & \multicolumn{1}{l|}{3409.1} & \multicolumn{1}{l|}{1.475} & \multicolumn{1}{l|}{469} & \multicolumn{1}{l|}{0.0031} & 119.8 \\ \hline
\multicolumn{1}{|l|}{NN} & \multicolumn{1}{l|}{0.0469} & \multicolumn{1}{l|}{1524.81} & \multicolumn{1}{l|}{Powell} & \multicolumn{1}{l|}{357.0} & \multicolumn{1}{l|}{175.9} & \multicolumn{1}{l|}{354.5} & \multicolumn{1}{l|}{171.3} & \multicolumn{1}{l|}{750.0} & \multicolumn{1}{l|}{358.4} & \multicolumn{1}{l|}{750.0} & \multicolumn{1}{l|}{451.9} & \multicolumn{1}{l|}{3535.8} & \multicolumn{1}{l|}{3.286} & \multicolumn{1}{l|}{809} & \multicolumn{1}{l|}{0.0041} & 149.1 \\ \hline
\multicolumn{1}{|l|}{NN} & \multicolumn{1}{l|}{0.0469} & \multicolumn{1}{l|}{1524.81} & \multicolumn{1}{l|}{SLSQP} & \multicolumn{1}{l|}{412.1} & \multicolumn{1}{l|}{243.8} & \multicolumn{1}{l|}{354.6} & \multicolumn{1}{l|}{194.9} & \multicolumn{1}{l|}{750.0} & \multicolumn{1}{l|}{499.9} & \multicolumn{1}{l|}{750.0} & \multicolumn{1}{l|}{500.0} & \multicolumn{1}{l|}{3610.0} & \multicolumn{1}{l|}{2.077} & \multicolumn{1}{l|}{545} & \multicolumn{1}{l|}{0.0038} & 169.3 \\ \hline
\multicolumn{1}{|l|}{NN} & \multicolumn{1}{l|}{0.0469} & \multicolumn{1}{l|}{1524.81} & \multicolumn{1}{l|}{Nelder-Mead} & \multicolumn{8}{l|}{optimum X found out of the bounds} & \multicolumn{1}{l|}{-} & \multicolumn{1}{l|}{5.672} & \multicolumn{1}{l|}{1600} & \multicolumn{1}{l|}{0.0035} & - \\ \hline
\multicolumn{1}{|l|}{GPR} & \multicolumn{1}{l|}{0.0037} & \multicolumn{1}{l|}{7473.84} & \multicolumn{1}{l|}{COBYLA} & \multicolumn{1}{l|}{320.1} & \multicolumn{1}{l|}{123.9} & \multicolumn{1}{l|}{320.5} & \multicolumn{1}{l|}{123.2} & \multicolumn{1}{l|}{621.1} & \multicolumn{1}{l|}{371.5} & \multicolumn{1}{l|}{730.8} & \multicolumn{1}{l|}{477.3} & \multicolumn{1}{l|}{4882.7} & \multicolumn{1}{l|}{23.225} & \multicolumn{1}{l|}{339} & \multicolumn{1}{l|}{0.0685} & 157.5 \\ \hline
\multicolumn{1}{|l|}{GPR} & \multicolumn{1}{l|}{0.0037} & \multicolumn{1}{l|}{7473.84} & \multicolumn{1}{l|}{Powell} & \multicolumn{1}{l|}{320.5} & \multicolumn{1}{l|}{124.6} & \multicolumn{1}{l|}{319.5} & \multicolumn{1}{l|}{123.5} & \multicolumn{1}{l|}{731.0} & \multicolumn{1}{l|}{478.4} & \multicolumn{1}{l|}{730.8} & \multicolumn{1}{l|}{477.9} & \multicolumn{1}{l|}{5082.1} & \multicolumn{1}{l|}{22.518} & \multicolumn{1}{l|}{314} & \multicolumn{1}{l|}{0.0717} & 10.3 \\ \hline
\multicolumn{1}{|l|}{GPR} & \multicolumn{1}{l|}{0.0037} & \multicolumn{1}{l|}{7473.84} & \multicolumn{1}{l|}{SLSQP} & \multicolumn{1}{l|}{320.1} & \multicolumn{1}{l|}{123.9} & \multicolumn{1}{l|}{320.5} & \multicolumn{1}{l|}{123.2} & \multicolumn{1}{l|}{621.1} & \multicolumn{1}{l|}{371.5} & \multicolumn{1}{l|}{730.8} & \multicolumn{1}{l|}{477.3} & \multicolumn{1}{l|}{4882.7} & \multicolumn{1}{l|}{11.241} & \multicolumn{1}{l|}{145} & \multicolumn{1}{l|}{0.0775} & 157.5 \\ \hline
\multicolumn{1}{|l|}{GPR} & \multicolumn{1}{l|}{0.0037} & \multicolumn{1}{l|}{7473.84} & \multicolumn{1}{l|}{Nelder-Mead} & \multicolumn{1}{l|}{320.5} & \multicolumn{1}{l|}{124.6} & \multicolumn{1}{l|}{319.5} & \multicolumn{1}{l|}{123.5} & \multicolumn{1}{l|}{731.0} & \multicolumn{1}{l|}{478.4} & \multicolumn{1}{l|}{730.8} & \multicolumn{1}{l|}{477.9} & \multicolumn{1}{l|}{5082.1} & \multicolumn{1}{l|}{66.400} & \multicolumn{1}{l|}{869} & \multicolumn{1}{l|}{0.0764} & 10.2 \\ \hline
\end{tabular}
\end{table*}

We build NN and GPR meta-models for this case study and compare the results generated using each of them. The meta-model parameters were tuned separately. 
%For the NN meta-model the number of hidden layers is fixed to 4, with 32 neurons in each layer. For the GPR model, we use an RBF kernel with length parameter 1 with bounds $(30,60)$. 
Out of the $4^8$ points, we select random subsets consisting of 25\%, 50\% and 75\% points as the training data, and the remaining points as test data to evaluate each meta-model. For each choice we report the extent of the match with the test points as a mean squared error (MSE), summarized in Table \ref{tab:obtained_results}. Python's \texttt{scikit-learn} library \cite{scikit-learn} was used for building the meta-models. 
%
The average computational time for building a GPR meta-model varied between 1000-7000 seconds depending on the training set size. For a NN model, the time for building and training the NN model varied from 300 to 1500 seconds. 
%
Once a meta-model is built, continuous, gradient-based optimizers can be applied over it to quickly identify promising regions in the design space. To arrive at a choice of optimizer for InventOpt, we consider multiple optimizers available in Python's SciPy library that are well-suited for this class of problems. The optimizers are listed in Table \ref{tab:obtained_results}. \textit{COBYLA} and \textit{Powell} are trust region-based optimization methods while \textit{SLSQP} and \textit{Nelder-Mead} are gradient-based methods and are popular choices for black-box optimization problems. We choose to use local optimizers rather than global optimizers because the idea is to utilize gradient information to quickly converge to regions of interest. One can use global optimizers such as Simulated Annealing which are more suited when the objective function is erratic or lacks trends. While this case study only considers local optimizers, we plan to incorporate a wider choice of optimizers in the InventOpt tool in future. Since the objective function may be non-convex we perform multiple optimization runs with random starting points for each optimizer.  Figure \ref{fig:convergence_plots} shows the convergence trends for each of these optimizers. Each subplot shows 20 optimization runs with distinct, randomly chosen starting points. The plot shows how the lowest $g(X)$ value found varies as the number of objective evaluations increases and presents a rough, visual estimate of the average convergence time for each optimizer. We observe that several runs in SLSQP and Powell converge faster,  though COBYLA and Nelder-Mead are resilient to noise. 
%
We observed that for the NN model the quality of results did not improve significantly with the size of training data, and the GPR model yielded considerably better results compared to the NN model. This may be because of the local minima inherently present in the NN model. The results for the GPR model are listed in Table \ref{tab:obtained_results} and summarized next.
%
%We identify the optimum reported by these results as a promising region and perform one more iteration. We narrowed the parameter ranges and restricted the search to a smaller region in the design space. Each parameter $S$ and $s$ are restricted to four equispaced values within these ranges. The maximum in this region is found at $X' = [S_{R1}=320, s_{R1}=130, S_{R2}=325, s_{R2}=125, S_{D1}=735, s_{D1}=480, S_{D2}=735, s_{D2}=475]$ with $f(X') = 3546.26$. 
The last column in Table \ref{tab:obtained_results} shows the Euclidean distance between the optimum found by an optimizer (choosing the best among the 20 independent runs) and $X'$ (the reference solution).  
%

%\end{landscape}
%The ranges for the parameters $S$ and $s$  are adjusted to $S_{R1} \in [310,325], s_{R1} \in [115,130], S_{R2} \in [310,325], s_{R2} \in [115,130], S_{D1} \in [720,735], s_{D1} \in [470,485], S_{D2} \in [720,735], s_{D2} \in [470,485]$.  We execute the simulation model with the input parameter values restricted to this region to identify the maximum within this constrained design space. It is observed to be present at $X' = [320, 130, 325, 125, 735, 480, 735, 475]$ with $f(X') = 3546.26$. The results were again visualised using the \textit{DataVis} tool. The meta-model-based optimization process was repeated for this new data set. However, the obtained results did not show any significant improvement in the optimum $X'$ found. The following section presents a comprehensive analysis of the results obtained from the case study and the insights derived from it.
%The present study employs the Bayesian Optimization method to estimate the optimal input parameter values that maximize the \textit{surplus} of the supply chain. Gaussian Process Regression (GPR) and Neural Network (NN) meta-models are utilized to obtain the posterior distribution via the application of the Bayes rule. Subsequently, various optimizers instead of acquisition functions are employed to determine the next promising region. A sampling of a limited number of points within the promising region is performed, followed by the application of the process of meta-model-based optimization. This iterative approach is repeated until the optimal solution within the design space is achieved. The process is detailed below.
%The optimizers are applied to both the GPR and NN meta-models, utilizing randomly sampled training datasets of 25\%, 50\%, and 75\% of the generated data. The results are summarized in \tablename~\ref{tab:obtained_results}. The highlighted cell indicates the optimized value determined by the optimizer, which is observed to be outside the predefined boundaries. Such found optimum is discarded. A performance discrepancy is observed between the Nelder-Mead optimizer and the NN meta-model. The column `optimum $g(X)$' refers to the optimum $P_{net}$ found by the optimizer on the respective meta-model. The next column lists the execution time taken for an optimizer to converge. The last column $d(X,X')$ lists the Euclidean distance between optimum $X$ found by the respective optimizer and the original optimum $X'$ mentioned below.
%The outcomes indicate that the maximum surplus values reported substantially deviate from the previously identified value of 3462.27, representing the maximum value in data obtained by running the simulation model. In addition, the reported optimal input parameter values exhibit considerable differences. These observations prompt redirecting the search for the optimum towards a more narrowly defined region, where a few additional points are sampled. This constitutes one iteration of the Bayesian optimization process.

\subsection{Results and Insights}
Observations from the meta-model based optimization case study show that the GPR model performs significantly better than the NN model in terms of the quality of solutions found. This could be attributed to the non-smooth approximation surface in the NN model. We observe that the quality of results obtained with the NN model do not show any significant improvement with respect to the training set size. However the NN model was found to be about 3x to 5x faster to train/build compared to the GPR model. A more exhaustive evaluation of NN models with other architectures needs to be performed in future. The GPR meta-model showed improved accuracy (indicated by the MSE with respect to the test data) with increasing training data-set sizes. The results for the GPR model are summarized in Table \ref{tab:obtained_results}. The GPR models provide a smoother surface for optimization and gradient-based optimizers such as Powell and COBYLA are found to perform well over it. 
%
\begin{figure}[!h]
  \vspace{-0.2cm}
  \centering
   %\includegraphics[scale=0.40]{image/supply_chain_nemon.png}
   {\epsfig{file = image/75gpr-convergence-graphs.png, width=0.42\textwidth}}
  %\caption{This figure illustrates the trade-off between the computational effort and the accuracy of the expected performance measure $P_{net}$.}
  \caption{Convergence plots for optimizers applied on the GPR meta-model.   The range of objective values $g(X)$ on the y-axis for all of the plots is from -5000 to -3000.}
  \label{fig:convergence_plots}
\end{figure}

The number of data points required for building the meta-model grows exponentially with the number of decision variables. Also, the computational time for fitting the meta-model grows with increasing training set size. We observe that the promising regions found by the GPR meta-models built with different training sizes align well, indicating that an incremental sampling approach to design space exploration can be used. A GPR meta-model can be initially built using a small number of training samples, optimizers could be applied over it to locate promising regions, and these regions could be explored further using fine-grained sampling, repeated in an iterative manner. 
%
Across the set of optimizers explored in this case study, we observed that the implementation of Nelder Mead in SciPy library does not accept parameter bounds as arguments since it is primarily intended for unconstrained optimization. This leads to several runs producing solutions that are outside acceptable bounds. Although there are ways to overcome this by modifying the objective function to introduce penalties for out-of-bound solutions, we observe that COBYLA, Powell and SLSQP methods can accept bounds and find good solutions.  These methods also show good convergence results. We observe that several runs of Powell and SLSQP converge to good solutions rapidly. 

In summary, this case study provides useful insights for design choices in the implementation of InventOpt and justifies the use of an iterative meta-model based approach for a similar class of problems in supply chain design and optimization. We plan to generalize and incorporate the steps performed in this case study into configurable, user-guided semi-automated routines in the InventOpt tool-set.
 
%Increasing the training data's size did not improve the obtained optimum for the NN meta-model. It performed poorly with the Nelder-Mead optimizer and yielded out-of-bound results. However, the average time to build a NN meta-model is much less than GPR. The NN meta-model with 50\% training data and Powell, SLSQP optimizers showed promising results and redirected the search to a promising region.

%The optimal value obtained by the GPR meta-model trained on 25\% of the generated data is consistent with that obtained by the GPR meta-model trained on 75\%. The optimum value $g(X)$ reported by GPR is $P_{net} \approx 5082$, which is higher than the actual optimum found in this region, which is approximately equal to $3546$.  In the case of the NN meta-model, the model with 50\% training samples shows good results where COBYLA, SLSQP and Powell converge to a point near the actual optimum. However, the Nelder-Mead optimization method does not require bounds, resulting in frequently obtaining solutions outside the acceptable range. This optimizer exhibited poor performance when applied to NN meta-models, yielding out-of-bounds results, which were discarded. The case study results indicate that the GPR meta-model outperforms NN for the given problem. However, it is worth noting that the computational cost of constructing the GPR meta-model is higher than NN. While GPR successfully directs the search for the optimum to the promising region, the reported optimum value differs from the actual one.

%We have considered a vast design space of $150^8$ design points for the case study, where each simulation takes 130 seconds. The computational cost can be huge when applying optimizers directly to the simulation model. However, deciding the number of sampling points, the meta-model, and its parameter values are critical for meta-model-based optimization. The case study demonstrates a systematic decision-making process facilitated by a visual interface. The insights in the case study can be used and applied to tackle similar problems. 

 %The GPR model is recommended when tackling inventory optimization problems in a supply chain as it performed better than the NN meta-model for the case study. The results show that utilizing 25\% of the data samples is sufficient for the GPR meta-model for leveraging the gradient information. The number of points to sample for building the meta-model is correlated to the design space's size. And GPR meta-model works well with the small subset of design space points as training points provided that the meta-model parameters are tuned to obtain a good fit over the original data points. The meta-model parameters can be adjusted iteratively by visualizing the fit in the \textit{DataVis} tool and accepting a fit with a reasonably small MSE. Powell and SLSQP local optimizers show improved performance with GPR meta-model due to their effective utilization of the gradient information exploited by the meta-model.



%The experiments showed that the GPR meta-model performs better than the NN meta-model. GPR shows the improved distance from the reference solution with increased training data size. It is likely due to the NN meta-model not always producing a smooth surface. However, GPR produces a smooth approximation surface to the training data, which aids in exploiting the gradient information for the optimizers.

%When considering sample size selection, the NN meta-model displayed relatively better outcomes when trained on 50\% of the samples. Conversely, the GPR meta-model showed improved performance as the training data size increased. Notably, the promising region reported by the GPR meta-model with 25\% training data aligns with the 75\% data. This indicates that a 25\% sample size is adequate for the GPR meta-model to identify suitable promising regions in our case.



%The smoothness of the approximation surface generated by GPR can be inferred from the similarity of the optimums reported by the GPR with 25\%, 50\% and 75\% training samples. It indicates that utilizing 25\% of the data samples is sufficient for the GPR meta-model for leveraging the gradient information and effectively directing the search for the optimal solution towards the promising region.
%In order to improve the accuracy of the meta-model, a larger number of points can be sampled to obtain a better fit. However, this approach would increase the time complexity of the overall optimization process. In the case study, random and regular grid sampling was utilized. However, other sampling techniques can be explored and experimented with to determine a better number of sample points. Choosing a meta-model for a given problem is a critical decision in the optimization process, and estimating the meta-model parameters is essential to obtain a good fit. 
%The optimization algorithms Powell and Nelder-Mead exhibited good performance with the GPR meta-model, while COBYLA performed well with the NN meta-model. It is necessary to conduct experiments with various optimizers when utilizing a particular meta-model, ultimately selecting the most effective one.


